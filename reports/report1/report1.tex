\documentclass[final,article,10pt]{scrartcl}
%
\usepackage{npac}
\usepackage{amsmath}
\usepackage{url}
\usepackage{graphicx}
\usepackage[ansinew]{inputenc}
\usepackage[T1]{fontenc} 
\usepackage[french]{babel}
\usepackage[fixlanguage]{babelbib}
\selectbiblanguage{french}

%
\begin{document}
%
\title{Comptes Rendus Projet 3I013}
\subtitle{N°1: Définition des premiers objectifs}
\author{Florian \textsc{Reynier} \& Mathis \textsc{Caristan}}
%
\date{}

\maketitle

\section*{Résumé des idées}
	Durant cette réunion nous avons pu commencer à fixer les premiers objectifs du projet qui sont donc les suivants:\\
    \par Dans un premier temps nous allons nous concentrer sur l'implantation de deux fonction, Une fonction déjà existante: xxhash et une fonction naïve conçue par nous même.\\
    A partir de ses fonctions, nous devrons effectuer une première analyse et présenter les résultats sous forme d'un histogramme. Pour effectuer ces tests, il est nécessaire de séléctionner un premier corpus généraliste qui sera composé de textes tirés du projet Gutemberg ainsi que de codes, car ce corpus se doit d'être le plus hétérogène possible. Nous chercherons à évaluer le comportement moyen de ses fontcions sur l'ensemble du corpus.\\
    \par Nous avons aussi discuté de manière plus générale à propos du projet et nous avons mis en évidence le paradoxe des anniversaires, représentatif du risque de collision. Il nous faudra donc dans le futur, être en mesure de décider d'une taille d'ensemble des valeurs hachage de 8, 16 ou 32 bits, en fonction de la taille des données en entrée.
	
	
	


%
% Non-BibTeX users please use



\end{document}
