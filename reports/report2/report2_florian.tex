\documentclass[final,twoside,article,10pt]{scrartcl}
%
\usepackage{npac}
\usepackage{amsmath}
\usepackage{url}
\usepackage{graphicx}
\usepackage[utf8]{inputenc}
\usepackage[T1]{fontenc} 
\usepackage[french]{babel}
\usepackage[fixlanguage]{babelbib}
\usepackage{hyperref}
\selectbiblanguage{french}

%
\begin{document}
%
\title{Compte Rendu Projet 3I013}
\subtitle{N°2: Récapitulatif du travail accompli
\author{Florian \textsc{Reynier} \& Mathis \textsc{Caristan}}
%
\date{09/02/2016}

\maketitle

\section{Résumé des idées de la réunion}
	
	Dans cette réunion il y a eu peu de changement sur les objectifs précédemment donnés nous continuons 
	donc sur le chemin que nous avons pris, notre principal tâche est de créer un histogramme des 
	performances des fonctions implantées puis de commencer à ajouter de nouvelles. 
	
\section{Création de notre fonction de hachage et implantation de XXHash}
\subsection*{monHash}
\subsection*{XXHash}

	La première tâche que j'ai entrepris a été de me documenter sur la fonction XXHash, la meilleure
	source d'information que j'ai pu trouver est une version en C de cette fonction bien documentée
	et complète.\\
	\paragraph Un premier problème de taille m'est alors apparu car XXhash est une fonction très 
	complexe, avec une optimisation très poussée allant jusqu'au moyen ou les données sont stockées
	en mémoire. Etant débutant en OCAML il m'a semblé impossible de recoder ces 1700 lignes de codes 
	complexe dans un langage que je maitrise peu pour le moment.\\
	\paragraph La soultion qui m'a semblé la meilleure a été de récupérer la fonction C utilisée
	comme documentation et de l'appeler dans une fonction OCAML. Cette fonction est opensource et
	peut etre trouvée ici:
	la fonction OCAML est définie comme suit:\\
	\"external\" précise que la fonction n'appartient pas à OCAML\\
	Le "main" de la fonction C a du être transformée au niveau de son type de retour et de ses retours.\\
	enfin les nombreux paramètres de la fonction, sont utilisable car la fonction OCAML transmet
	le pointeur sur son tableau d'argument, ceux ci peuvent donc être facilement récupéré\\
	En conclusion on a donc une fonction utilisable en OCAML aussi complète que son original en C
	pour un coût très réduit en temps.
%
% Non-BibTeX users please use



\end{document}
