\documentclass[final,twoside,article,10pt]{scrartcl}
%
\usepackage{npac}
\usepackage{amsmath}
\usepackage{url}
\usepackage{graphicx}
\usepackage[utf8]{inputenc}
\usepackage[T1]{fontenc} 
\usepackage[french]{babel}
\usepackage[fixlanguage]{babelbib}
\selectbiblanguage{french}

%
\begin{document}
%
\title{Compte Rendu de Projet de recherche de L3}
\subtitle{N\degre 2 : Explications du raisonnement.}
\author{Florian \textsc{Reynier} \& Mathis \textsc{Caristan}}
%
\date{11/02/2016}

\maketitle

\section{Explication de la création de notre fonction de hachage}
    La création de la fonction de hachage a été réalisée en s'aidant des remarques du livre \emph{Types de données et algorithmes} \cite{soria}.\\
    La construction de la fonction de hachage a été faite en plusieurs étapes. Nous avons dans un premier temps décidé d'associer à chaque caractère $c$, une valeur $v_{|10}$. Puis, cette valeur $v_{|10}$ a été convertie en binaire pour obtenir $v_{|2}$. Un mot $m$ était donc représenté par une suite $S$ d'élément binaires. Pour contracter cette suite $S$, nous avons choisi de la découper en "sous-suites" $s_i$ de longeur $l$, et d'appliquer une opération binaire $o$ entres les $s_i$, pour obtenir une unique suite $s_f$ de taille $l$. Cette suite était alors reconvertie en décimale pour obtenir la valeur $e$. Enfin, la dernière étape a consistée à appliquer la fonction $f_{mult}$ à $e$ (voir équation \ref{eq:mult}), afin d'obtenir $f_{mult}(e) = h$, la valeur de hachage du mot.
    \begin{equation}
        \label{eq:mult}
        f_{mult}(e) = \lfloor ( (e*\theta) \mod 1) * T \rfloor
    \end{equation}
    \subsection{Associer une valeur $v_{|10}$ à un caractère}
    Comme expliqué plus haut, la première étape a été d'associé une valeur à chaque caractère. Pour cela, nous avions plusieurs options, la première à laquelle nous avons pensé, était d'utiliser la table ASCII. Cependant, cette méthode présente plusieurs défauts. Tout d'abord, la taille de la table ASCII est trop grande pour ce dont nous avons besoin ici, en effet, nous ne considérons que les lettres (un mot commme "\texttt{(texte)}" ne doit pas être hashé en prenant en compte les parenthèses). De plus, la table ASCII fait la différence entre les majuscules et les minuscules, ce que nous ne cherchont pas non plus. Enfin, le dernier problème est dû à la représentation des lettres accentuées en OCaml. En effet, celles-ci ne sont pas considérées comme un unique caractère, mais comme une chaîne de deux caractères spéciaux : $\texttt{é} = "\backslash 195\backslash 169"$. 
    Du coup, nous avons plutôt choisi d'associer à chaque lettre, sa place dans l'alphabet. De plus, en considérant les lettres accentuées comme non accentuées (un \texttt{é} est équivalent à un \texttt{e}), cela nous permet de résoudre le problème de l'accentuation en OCaml. Cette méthode a également l'avantage de nous laisser libre de rajouter aisément des caractères si nous le souhaitons (notamment les lettres accentuées).

    \subsection{Choix de l'opération primaire de la fonction de hachage}
    Dans \cite{soria}, les opérations primaires présentées sont \textbf{l'extraction}(on ne considère arbirtrairement que certains bits d'une représention binaire d'un mot), \textbf{la compression} (voir \ref{sec:compression}), \textbf{la division} (reste d'une division entière d'une représentation décimale d'un mot) et \textbf{la multiplication} (modulo 1 de la multiplication d'une représentation décimale d'un mot par un paramètre).
    Les première opération ne donne pas de bons résultats en terme de hachage, tandis que la seconde est généralement surtout utilisée pour réduire la taille d'une chaîne de bits. Les deux dernières semblent être des bonnes opérations primaires, et nous avons choisi de retenir la multiplication pour notre fonction de hachage. Le paramètre multiplicatif $\theta$ a été choisi parmi deux valeurs qui donnent théoriquement la meilleure uniformité de distribution de clefs (\cite{soria}).

    \subsection{Compression}
    \label{sec:compression}
    Lorsque nous avons cherché quelle valeur $e$ associer à un mot pour la multiplication, nous avons envisagé plusieurs possibilités. La simple addition, ou multiplication, des valeurs $v_{|10}$ des lettres composant un mot peut créer trop facilement des collisions dans la table de hachage.
    Nous avons donc choisi à la place de représenter un mot par la "juxtaposition des valeurs de ses caractères" ($"\texttt{ABCDE}" = 12345$).
    Cependant, cette méthode peut facilement créer de très grandes valeurs, voire dépasser la capacité d'un \textsc{int} dans certains cas.
    Pour palier à ce problème, nous avons choisi de convertir les valeurs $v_{|10}$ des caracères en binaire (sur 5bits), et de représenter un mot comme un tableau de bits ($"\texttt{A B C D E}" = 1-2-3-4-5 = 00001_{|2}-00010_{|2}-00011_{|2}-00100_{|2}-00101_{|2}$, donc $"\texttt{ABCDE}" = 0000100010000110010000101$).
    Cette nouvelle représentation assure de manière unique la représentation de chaque mot, cependant, la taille de cette ensemble est trop grande ($2^{5*n}$ pour un mot de n lettres). 
    Pour une table de hachage de taille $2^{16}$, il n'est pas intéressant d'avoir plus de 16bits.
    Ainsi, nous avons cherché à compresser cette longue chaîne de bits à une plus courte.
    Pour cela, nous avons découpé la chaîne $S$ d'un mot, en $s_i$ sous chaînes de 16bits. Ensuite, en appliquant une opération binaire entre ces sous-chaînes, nous obtenons une unique chaîne finale $s_f$.
    Les opérations \textbf{AND} et \textbf{OR} ont la mauvaise propriété d'entraîner des accumulations en début et en fin de tableau respectivement, tandis que l'opération \textbf{XOR} (ou exclusif) semble être plus équitable.
    Une fois que nous avons obtenu cette chaîne de 16 bits, nous pouvons la reconcertir en décimal pour obtenir la valeur $e$ du mot avec lequel nous travaillons à l'étape de multiplication.\newline
    \medskip\newline
    \textbf{NB} : C'est lors de l'application de l'opération \textbf{XOR} que sont crées les collisions dans cette méthode. En effet, un mot correspond à une unique chaîne $S$ et réciproquement, mais ce n'est plus nécessairement le cas après l'opération.
    \begin{thebibliography}{9}
    \bibitem{soria}
        Marie-Claude Gaudel, Michèle Sorian Christine Froidevaux, \emph{Types de données et algorithmes}. Ediscience internationnal, Paris, 1993.
\end{thebibliography}

\end{document}
