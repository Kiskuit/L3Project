\documentclass[final,twoside,article,10pt]{scrartcl}
%
\usepackage{npac}
\usepackage{amsmath}
\usepackage{url}
\usepackage{graphicx}
\usepackage[utf8]{inputenc}
\usepackage[T1]{fontenc} 
\usepackage[french]{babel}
\usepackage[fixlanguage]{babelbib}
\selectbiblanguage{french}

%
\begin{document}
%
\title{Compte Rendu de Projet de recherche de L3}
\subtitle{N\degre 2 : Explications du raisonnement.}
\author{Florian \textsc{Reynier} \& Mathis \textsc{Caristan}}
%
\date{11/02/2016}

\maketitle

\section{Explication de la création de notre fonction de hashage}
    La création de la fonction de hachage a été réalisée en s'aidant des remarques du livre \emph{Types de données et algorithmes} (\ref{soria}).\\
    La construction de la fonction de hachage a été faite en plusieurs étapes. Nous avons dans un premier temps décidé d'associer à chaque caractère $c$, une valeur $v_{|10}$. Puis, cette valeur $v-{|10}$ a été convertie en binaire pour obtenir $v_{|2}$. Un mot $m$ était donc représenté par une suite $S$ d'élément binaires. Pour contracter cette suite $S$, nous avons choisi la découper en "sous-suite" $s$ de longeur $l$, et d'appliquer une opération binaire $o$ entres les $s$, pour obtenir une unique suite $s_f$ de taille $l$. Cette suite était alors reconvertie en décimale pour obtenir la valeur $e$. Enfin, la dernière étape a consistée à appliquer la fonction suivante $f_{mult}$ à $e$, afin d'obtenir $f_{mult}(e) = h$, la valeur de hashage du mot.
    \subsection{Associer une valeur à un charactère}
    \begin{equation}
        f_{mult}(e) = \lfloor ( (e*\theta) \mod 1) * T \rfloor
    \end{equation}

    \begin{thebibliography}
    \bibitem{soria}
        Marie-Claude Gaudel, Michèle Sorian Christine Froidevaux, \emph{Types de données et algorithmes}. Ediscience internationnal, Paris, 1993.
\end{thebibliography}

\end{document}
